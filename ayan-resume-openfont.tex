%%%%%%%%%%%%%%%%%%%%%%%%%%%%%%%%%%%%%%%
% Deedy - One Page Two Column Resume
% LaTeX Template
% Version 1.2 (16/9/2014)
%
% Original author:
% Debarghya Das (http://debarghyadas.com)
%
% Original repository:
% https://github.com/deedydas/Deedy-Resume
%
% IMPORTANT: THIS TEMPLATE NEEDS TO BE COMPILED WITH XeLaTeX
%
% This template uses several fonts not included with Windows/Linux by
% default. If you get compilation errors saying a font is missing, find the line
% on which the font is used and either change it to a font included with your
% operating system or comment the line out to use the default font.
% 
%%%%%%%%%%%%%%%%%%%%%%%%%%%%%%%%%%%%%%
% 
% TODO:
% 1. Integrate biber/bibtex for article citation under publications.
% 2. Figure out a smoother way for the document to flow onto the next page.
% 3. Add styling information for a "Projects/Hacks" section.
% 4. Add location/address information
% 5. Merge OpenFont and MacFonts as a single sty with options.
% 
%%%%%%%%%%%%%%%%%%%%%%%%%%%%%%%%%%%%%%
%
% CHANGELOG:
% v1.1:
% 1. Fixed several compilation bugs with \renewcommand
% 2. Got Open-source fonts (Windows/Linux support)
% 3. Added Last Updated
% 4. Move Title styling into .sty
% 5. Commented .sty file.
%
%%%%%%%%%%%%%%%%%%%%%%%%%%%%%%%%%%%%%%%
%
% Known Issues:
% 1. Overflows onto second page if any column's contents are more than the
% vertical limit
% 2. Hacky space on the first bullet point on the second column.
%
%%%%%%%%%%%%%%%%%%%%%%%%%%%%%%%%%%%%%%


\documentclass[]{deedy-resume-openfont}
\usepackage{fancyhdr}
 
\pagestyle{fancy}
\fancyhf{}
 
\begin{document}

%%%%%%%%%%%%%%%%%%%%%%%%%%%%%%%%%%%%%%
%
%     LAST UPDATED DATE
%
%%%%%%%%%%%%%%%%%%%%%%%%%%%%%%%%%%%%%%
\lastupdated

%%%%%%%%%%%%%%%%%%%%%%%%%%%%%%%%%%%%%%
%
%     TITLE NAME
%
%%%%%%%%%%%%%%%%%%%%%%%%%%%%%%%%%%%%%%
\namesection{Ayan}{Sinha Mahapatra}{ \urlstyle{same}\href{https://ayansinhamahapatra.github.io/}{ayansinhamahapatra.github.io}\\
+919674426869 | \href{mailto:ayansmahapatra@gmail.com}{ayansmahapatra@gmail.com}
}

%%%%%%%%%%%%%%%%%%%%%%%%%%%%%%%%%%%%%%
%
%     COLUMN ONE
%
%%%%%%%%%%%%%%%%%%%%%%%%%%%%%%%%%%%%%%

\begin{minipage}[t]{0.33\textwidth} 

%%%%%%%%%%%%%%%%%%%%%%%%%%%%%%%%%%%%%%
%     EDUCATION
%%%%%%%%%%%%%%%%%%%%%%%%%%%%%%%%%%%%%%

\section{Education} 

\subsection{Jadavpur University}
\descript{B.E. in Electronics And TeleCommunications Engg. }
\location{May 2016 - Present | Kolkata, India}
\location{ Cum. CGPA: 8.1 / 10 }
\sectionsep

\subsection{Bidhannagar High School}
Higher Secondary Exam \\
\location{ May 2016 |  Kolkata, India}
\location{Marks Percentage 87.40 }
\sectionsep

\subsection{RKMV Purula}
Secondary Exam \\
\location{ May 2014 |  Purulia, India}
\location{Marks Percentage 91.90 }
\sectionsep

%%%%%%%%%%%%%%%%%%%%%%%%%%%%%%%%%%%%%%
%     LINKS
%%%%%%%%%%%%%%%%%%%%%%%%%%%%%%%%%%%%%%

\section{Links} 
Github:// \href{https://github.com/ayansinhamahapatra}{\bf ayansinhamahapatra} \\
LinkedIn://  \href{https://www.linkedin.com/in/ayansinhaju/}{\bf Ayan Sinha Mahapatra} \\
Twitter://  \href{https://twitter.com/ayansm23}{\bf @ayansm23} \\

%%%%%%%%%%%%%%%%%%%%%%%%%%%%%%%%%%%%%%
%     COURSEWORK
%%%%%%%%%%%%%%%%%%%%%%%%%%%%%%%%%%%%%%

\section{Coursework}

\subsection{Undergraduate}
Data Structures And Algorithms \\
C Programming And Numerical Analysis \\
Computer Architecture \\
Microrpocessors And Microcontrollers + Practicum \\
Control Systems + Practicum \\
Analog Systems + Practicum \\
Digital Logic + Practicum \\
Analog and Digital Comm. + Practicum \\ 
\sectionsep

\subsection{MOOC Coursera}
Neural Networks \\
Machine Learning \\
Computational Neuroscience \\
Deep Learning Specialization \\
Data Science with Python Specialization \\

%%%%%%%%%%%%%%%%%%%%%%%%%%%%%%%%%%%%%%
%     SKILLS
%%%%%%%%%%%%%%%%%%%%%%%%%%%%%%%%%%%%%%

\section{Skills}
\subsection{Programming}
\location{Over 1000 lines:}
C \textbullet{} Python \textbullet{} Matlab \\
\location{Familiar:}
HTML \textbullet{} Lua \textbullet{} Assemmbly \\
\location{C Libraries:}
Eigen \textbullet{} MLPack \\
\location{Python Libraries:}
Tensorflow \textbullet{} Keras \textbullet{} NumPy \textbullet{} Pandas \textbullet{} Matplotlib \textbullet{} Scikit-Learn \textbullet{} OpenCV \textbullet{} Django \textbullet{} Flask
\sectionsep

%%%%%%%%%%%%%%%%%%%%%%%%%%%%%%%%%%%%%%
%
%     COLUMN TWO
%
%%%%%%%%%%%%%%%%%%%%%%%%%%%%%%%%%%%%%%

\end{minipage} 
\hfill
\begin{minipage}[t]{0.66\textwidth} 

%%%%%%%%%%%%%%%%%%%%%%%%%%%%%%%%%%%%%%
%    Projects
%%%%%%%%%%%%%%%%%%%%%%%%%%%%%%%%%%%%%%

\section{Projects}
\runsubsection{Signature Verification and Full Stack Developement}
\descript{| Hackathon Participant }
\location{Nov 2017 - July 2018 | Mumbai, India}
With \textbf{\href{https://github.com/Sayan98}{Sayan Goswami}}, I've participated in the SBI Hackathon 2017, where we successfully developed a Signature Verification System (Using Siamese ConvNets) and Analytics Dashboard/Cloud Hosting. We participated in a Proof Of Concept Demonstration at SBI Global IT Center, Mumbai and went on to win the Hackathon. \textbf{\href{https://github.com/AyanSinhaMahapatra/AutoSIGN}{ [GitHub Link] }}
\sectionsep

\runsubsection{Cubesolving}
\descript{| Personal Project }
I was working on a Cube Solving Program (without any external standard Cube Libraries, i.e. from scratch). \textbf{\href{https://github.com/AyanSinhaMahapatra/CubeSolving}{ [GitHub Link] }}
\sectionsep

\runsubsection{Forum Website}
\descript{| Personal Project }
I worked on a Website Development Project Using Python-Django with Full Forum/Social Network like features for a University Club. 
\textbf{\href{https://github.com/AyanSinhaMahapatra/Univnet_Beta}{ [GitHub Link 1] }} \textbf{\href{https://github.com/AyanSinhaMahapatra/AnswerIt}{ [GitHub Link 2] }}
\sectionsep

%%%%%%%%%%%%%%%%%%%%%%%%%%%%%%%%%%%%%%
%     RESEARCH
%%%%%%%%%%%%%%%%%%%%%%%%%%%%%%%%%%%%%%

\section{Research}
\runsubsection{JU Artificial Intelligence Lab}
\descript{| Ungergraduate Researcher}
\location{Nov 2018 – Present | Kolkata, India}
Working under \textbf{\href{https://www.amitkonar.com/}{Prof Amit Konar}} on applications of Fuzzy Random Forest Algorithms for Circuit/Plant Fault Detection. I'll be working on Neuroscience problems from this January. 
\sectionsep

\runsubsection{ISI Kolkata ECSU Dept.}
\descript{| Undergraduate Researcher}
\location{Dec 2017 – Present | Kolkata, India}
I have been working under the guidance of \textbf{\href{https://www.isical.ac.in/~swagatam.das/}{Prof Swagatam Das}}. The Research Problem is Based on applications of Evolutionary Algorithms (a VNS variant of Genetic Algorithms) on an Integer Programming problem \textbf{\href{https://github.com/AyanSinhaMahapatra/Gene_Algos}{ [GitHub Link] }}. I've also worked on advancements in Differential Evolution Algorithms (specifically SHADE/L-SHADE).

%%%%%%%%%%%%%%%%%%%%%%%%%%%%%%%%%%%%%%
%     ACHEIVEMENTS
%%%%%%%%%%%%%%%%%%%%%%%%%%%%%%%%%%%%%%

\section{Achievements} 
\begin{tabular}{rll}
2018	     & Winning team/~1500 teams  & SBI Automate For Bank Hackathon 2017\\
2017	     & 4th Team in College  & ICPC Online Round\\
2016	     & Ranked 306  & WBJEE\\
\end{tabular}
\sectionsep

%%%%%%%%%%%%%%%%%%%%%%%%%%%%%%%%%%%%%%
%     Clubs And Societies
%%%%%%%%%%%%%%%%%%%%%%%%%%%%%%%%%%%%%%

\section{Clubs and Societies}
 
\runsubsection{TEDxJadavpurUniversity}
\descript{| Tresurer and OC Member}
Worked as an organizer at the 2018 Edition of TEDxJadapurUniversity, and in the senior team for the 2019 edition.  
\sectionsep

\runsubsection{AI and Kaggle University Club}
\descript{| Co-Founder}
Co-founded the club with my classmate \textbf{\href{https://github.com/Sayan98}{Sayan Goswami}}. \\
\sectionsep 

\runsubsection{Robotics Club}
\descript{| Volunteer }
Organized an Image Processing/Machine Learning Workshop, participated in Robotics Competitions. 
\sectionsep


\end{minipage} 
\end{document}  \documentclass[]{article}
